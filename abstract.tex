\documentclass[10pt,oneside]{article}

%%%%%%%%%%%%%
\setlength{\textheight}{8.75in} %Letter is 11in, less 2 for margins, less 0.25 for footer
\setlength{\oddsidemargin}{0.0in} %gets +1inc
\setlength{\evensidemargin}{0.0in} %gets +1inch
\setlength{\textwidth}{6.50in} %Letter is 8.5, less 2 inches for margins
\setlength{\topmargin}{0.0in}
\setlength{\headheight}{0in}
\setlength{\headsep}{0in}
\setlength{\parindent}{0.25in}
%%%%%%%%%%%% 

\usepackage[numbers]{natbib}
\usepackage{graphicx}
\usepackage[colorlinks=true,citecolor=black,urlcolor=blue]{hyperref}

\title{BioJava Project Update 2016}
\author{\small \underline{Michael~Heuer}$^{1}$, Paolo Pavan$^{2}$,
\and \small Additional Author$^{2}$, Additional Author${^2}$}
\date{}

\begin{document}

\footnotetext[1]{AMPLab, University of California, Berkeley, CA}
\footnotetext[2]{Genomnia srl, Bresso (MI), Italy}
\footnotetext[3]{Additional Affiliation, Springfield, USA}

\renewcommand{\thefootnote}{\fnsymbol{footnote}}

\footnotetext{Correspondence should be addressed to heuermh@berkeley.edu.}

\maketitle

\vspace{-0.25in}
\noindent
{\small
Website: \url{http://biojava.org} \\
Repositories: \url{https://github.com/biojava/biojava} and \url{https://github.com/biojava/biojava-legacy}\\
License: GNU Lesser General Public License (LGPL)\\
}

BioJava~\cite{prlic2012biojava} is a mature open-source project that provides a framework for processing of
biological data. BioJava contains powerful analysis and statistical routines, tools for parsing common file
formats, and packages for manipulating sequences and 3D structures. It enables rapid bioinformatics application
development in the Java programming language. The BioJava project is supported by the Open Bioinformatics
Foundation (OBF).


BioJava version 4.2.0 was released in March 2016, with over 750 commits from 10 contributors since the
previous release.  Several new features were added, such as a full DSSP--compatible implementation of secondary
structure assignment, improvements in symmetry detection, a unified StructureIdentifier framework, and a
new SearchIO framework to interface {BLAST} (or {BLAST}--like) searches.  Macromolecular Crystallographic
Information File ({mmCIF}) support was improved, including parsing of bonds, sites, and charges for
chemical components.


Since version 4.2.0, work is progressing towards a new major release version 5.0.0.  The minimum JDK has been
bumped to JDK8 to allow access to new language syntax and features.  Compound has been refactored to entity
and now considers non-polymer compounds.  Support for a new Macromolecular Transmission Format ({MMTF}), a compact
binary format to represent biomolecular structures from Protein Data Bank ({PDB}), is under active development.

\bibliographystyle{abbrv}
\bibliography{abstract}

\end{document}
